\documentclass[a4paper, 16pt]{extreport}

\usepackage{graphicx}
\usepackage[T2A]{fontenc}
\usepackage[utf8]{inputenc}
\usepackage[english, russian]{babel}
\usepackage{mathtools}


\title  {
	Разработка параллельной версии программы сложения перемноженных	матриц}

\author {Озерова Татьяна Александровна}

\begin{document}
	\maketitle
	\tableofcontents{}
	
 	\chapter*{Постановка задачи }
 	\makeatletter
 	\renewcommand\chapter{\par
 		\thispagestyle{plain}
 		\@afterindentfalse \secdef\@chapter\@schapter} 
 	Требуется реализовать программу с использованием OpenMP, реализующая сложение матриц, первая из которых является матрицей, умноженная на const $\beta$, вторая - результат перемножения матриц A и B, умноженный на const $\alpha$. 
 	
 	Формула  для вычисления:
 	
 	\centerline{Matrix = C*$\beta$ + A*B*$\alpha$,}
 	
 	 
 	 где \\ C = $\begin{pmatrix}
 		c_{1,1} & \cdots & c_{1,m} \\
 		c_{2,1} & \cdots & c_{2,m} \\
 		\vdots & \ddots & \vdots  is\\
 		c_{n,1} & \cdots & c_{n,m} 
 	\end{pmatrix}$,
 	A  = $\begin{pmatrix}
 		a_{1,1}  & \cdots & a_{1,k} \\
 		a_{2,1} & \cdots & a_{2,k} \\
 		\vdots  & \ddots & \vdots  \\
 		a_{n,1} & \cdots & a_{n,k} 
 	\end{pmatrix}$, 
 	B = $\begin{pmatrix}
 		b_{1,1}  & \cdots & b_{1,m} \\
 		b_{2,1} & \cdots & b_{2,m} \\
 		\vdots   & \ddots & \vdots  \\
 		b_{k,1} & \cdots & b_{k,m} 
 	\end{pmatrix}$  
    n,m,k - параметры, задающие размеры матриц 
    \chapter*{Результаты}
    Результаты измерений представлены в таблице, по которой был построен график.
    \\\\\\
    $\begin{tabular}{ | l | l | l | l | l | l | }
    	\hline
    	Процессы p & 64 &  128 & 256 & 512 & 1024 & 2048 \\ 
    	\hline
    	4 & 0.000384485 & 0.0168771& 0.134175 & 1.07203&8.53774&309.695 \\
    	8 &0.000197584& 0.00848731&0.067121&0.53596&4.26882&155.456\\
    	16&0.000108214&0.00424143&0.0336281&0.268785& 2.18194&77.7303\\
    	64&4.28729e-05&0.00108419&0.00854691&0.0674516&0.553936&20.0047\\
    	128&4.26494e-05&0.000546804&0.00428574&0.0342073&0.277617&9.88401\\
    	256&4.37294e-05&0.000547071&0.00215037&0.0171252&0.140187&4.98058\\
       	\hline
    \end{tabular}$

	\includegraphics[scale = 0.3]{gr.png}
    \chapter*{Выводы}
    \begin{itemize}
    	\item Разработана параллельная версия алгоритма сложения перемноженных матриц 
    	\item С увеличением количества ядер время на выполнение алгоритма может так же увеличиваться
    	
    \end{itemize}
	
\end{document}
